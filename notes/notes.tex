\documentclass[12pt]{article}

\usepackage{fouriernc}
\usepackage[T1]{fontenc}
\usepackage{amsmath}
\usepackage{amssymb}
\usepackage[utf8]{inputenc}
\usepackage[english]{babel}
\usepackage{multicol}
\usepackage[margin=1in]{geometry}

\newcommand{\curly}   [1]{\left\{      #1 \right\}}
\newcommand{\round}   [1]{\left(       #1 \right)}
\newcommand{\hard}    [1]{\left[       #1 \right]}
\newcommand{\straight}[1]{\left|       #1 \right|}
\newcommand{\ceiling} [1]{\left\lceil  #1 \right\rceil}
\newcommand{\floor}   [1]{\left\lfloor #1 \right\rfloor}

\begin{document}

\section*{Fundamentals}

\subsection*{Evaluating a polynomial}
\begin{align*}
    P(x) &= 4x^5 + 7x^8 - 3x^{11} + 2x^{14} \\
    \\
    x^2 &= x \times x \\
    x^4 &= x^2 \times x^2 \\
    x^5 &= x^4 \times x \\
    x^8 &= x^4 \times x^4 \\
    x^{10} &= x^8 \times x^2 \\
    x^{11} &= x^{10} \times x \\
    x^{14} &= x^{10} \times x^4
\end{align*}
Evaluating $P(x)$ requires 7 multiplications and 3 additions. As an alternative writing,
\begin{align*}
    P(x) &= 4x^5 + 7x^8 - 3x^{11} + 2x^{14} \\
    &= x^5 \times (4 + x^3 \times (7 + x^3 \times (-3 + x^3 \times 2))) \\
    \\
    x^2 &= x \times x \\
    x^3 &= x^2 \times x \\
    x^5 &= x^3 \times x^2
\end{align*}

\subsection*{Floating point}
\begin{align*}
    \sqrt{a} - b &= \frac{a - b^2}{\sqrt{a} + b}
\end{align*}

Absolute error: $\straight{\text{Exact} - \text{Rounded}}$
Relative error: $\frac{\text{Absolute error}}{\text{Exact solution}}$

\subsection*{Intermediate Value Theorem}
Given the continuous function $f$ and $y \in \mathbb{R}$ such that $f(a) \leq y \leq f(b)$, on the interval $[a, b]$, there exists $c \in \mathbb{R}$ such that $f(c) = y$.

\subsection*{Taylor Polynomials}
$i$-term Taylor polynomial:
\begin{align*}
    \sum_{i=0}^n \frac{f^{(i)}(x_0) \times (x - x_0)}{i!}
\end{align*}
Maximum error term:
\begin{align*}
    \frac{f^{(n)}(c) \times (x - x_0)^i}{n!}
\end{align*}
Choose $c$ and $x$ in the interval to maximize the error.

\section*{Solving Equations}

\begin{center}
    \begin{tabular}{l|lc}
        Method & Convergence & Bracketed \\ \hline
        Bisection & $\frac{e_{i+1}}{e_i} = \frac{1}{2}$ & Yes \\
        Fixed-point & $\frac{e_{i+1}}{e_i} = \straight{g'(r)}$ & No
    \end{tabular}
\end{center}

\subsection*{Bisection}
\begin{align*}
    \epsilon &< \frac{b - a}{2^{n + 1}} \\
    n &= \ceiling{\log_2 \round{\frac{b - a}{\epsilon} - 1}}
\end{align*}

\end{document}