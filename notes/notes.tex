\documentclass[12pt]{article}

\usepackage{fouriernc}
\usepackage[T1]{fontenc}
\usepackage{amsmath}
\usepackage{amssymb}
\usepackage[utf8]{inputenc}
\usepackage[english]{babel}
\usepackage{multicol}
\usepackage[margin=1in]{geometry}

\newcommand{\curly}[1]{\left\{ #1 \right\}}
\newcommand{\round}[1]{\left( #1 \right)}
\newcommand{\hard}[1]{\left[ #1 \right]}
\newcommand{\C}[2]{
    \begin{pmatrix}
        #1 \\ #2
    \end{pmatrix}
}

\begin{document}

\section*{Fundamentals}

\subsection*{Evaluating a polynomial}
\begin{align*}
    P(x) &= 4x^5 + 7x^8 - 3x^{11} + 2x^{14} \\
    \\
    x^2 &= x \times x \\
    x^4 &= x^2 \times x^2 \\
    x^5 &= x^4 \times x \\
    x^8 &= x^4 \times x^4 \\
    x^{10} &= x^8 \times x^2 \\
    x^{11} &= x^{10} \times x \\
    x^{14} &= x^{10} \times x^4
\end{align*}
Evaluating $P(x)$ requires 7 multiplications and 3 additions. As an alternative writing,
\begin{align*}
    P(x) &= 4x^5 + 7x^8 - 3x^{11} + 2x^{14} \\
    &= x^5 \times (4 + x^3 \times (7 + x^3 \times (-3 + x^3 \times 2))) \\
    \\
    x^2 &= x \times x \\
    x^3 &= x^2 \times x \\
    x^5 &= x^3 \times x^2
\end{align*}

\subsection*{Floating point}
\begin{align*}
    \sqrt{a} - b &= \frac{a - b^2}{\sqrt{a} + b}
\end{align*}

\subsection*{Intermediate Value Theorem}
Given the continuous function $f$ and $y \in \mathbb{R}$ such that $f(a) \leq y \leq f(b)$, on the interval $[a, b]$, there exists $c \in \mathbb{R}$ such that $f(c) = y$.

\section*{Solving Equations}

\subsection*{The Bisection Method}

\end{document}