\documentclass[12pt]{article}

\usepackage{fouriernc}
\usepackage[T1]{fontenc}
\usepackage{amsmath}
\usepackage{amssymb}
\usepackage[utf8]{inputenc}
\usepackage[english]{babel}
\usepackage{multicol}
\usepackage[margin=1in]{geometry}

\newcommand{\curly}   [1]{\left\{      #1 \right\}}
\newcommand{\round}   [1]{\left(       #1 \right)}
\newcommand{\hard}    [1]{\left[       #1 \right]}
\newcommand{\straight}[1]{\left|       #1 \right|}
\newcommand{\ceiling} [1]{\left\lceil  #1 \right\rceil}
\newcommand{\floor}   [1]{\left\lfloor #1 \right\rfloor}

\begin{document}

\section*{Fundamentals}

\subsection*{Evaluating a polynomial}
\begin{align*}
    P(x) &= 4x^5 + 7x^8 - 3x^{11} + 2x^{14} \\
    \\
    x^2 &= x \times x \\
    x^4 &= x^2 \times x^2 \\
    x^5 &= x^4 \times x \\
    x^8 &= x^4 \times x^4 \\
    x^{10} &= x^8 \times x^2 \\
    x^{11} &= x^{10} \times x \\
    x^{14} &= x^{10} \times x^4
\end{align*}
Evaluating $P(x)$ requires 7 multiplications and 3 additions. As an alternative writing,
\begin{align*}
    P(x) &= 4x^5 + 7x^8 - 3x^{11} + 2x^{14} \\
    &= x^5 \times (4 + x^3 \times (7 + x^3 \times (-3 + x^3 \times 2))) \\
    \\
    x^2 &= x \times x \\
    x^3 &= x^2 \times x \\
    x^5 &= x^3 \times x^2
\end{align*}

\subsection*{Floating point}
\begin{align*}
    \sqrt{a} - b &= \frac{a - b^2}{\sqrt{a} + b}
\end{align*}

Absolute error: $\straight{\text{Exact} - \text{Rounded}}$
Relative error: $\frac{\text{Absolute error}}{\text{Exact solution}}$

\subsection*{Intermediate Value Theorem}
Given the continuous function $f$ and $y \in \mathbb{R}$ such that $f(a) \leq y \leq f(b)$, on the interval $[a, b]$, there exists $c \in \mathbb{R}$ such that $f(c) = y$.

\subsection*{Taylor Polynomials}
$i$-term Taylor polynomial:
\begin{align*}
    P_3 &= \sum_{i=0}^n \frac{f^{(i)}(x_0) \times (x - x_0)}{i!}
\end{align*}
Maximum error term:
\begin{align*}
    E_n &= \frac{f^{(n)}(c) \times (x - x_0)^i}{n!}
\end{align*}
Choose $c$ and $x$ in the interval to maximize the error. The error term is the next n after the order of the Taylor polynomial.

\section*{Solving Equations}

% TODO: Add next step formula to this table
% Bottom doggy eared page with this
\begin{center}
    \begin{tabular}{l|lc}
        Method & Convergence & Bracketed \\ \hline
        Bisection & $\frac{e_{i+1}}{e_i} = \frac{1}{2}$ & Yes \\
        Fixed-point & $\frac{e_{i+1}}{e_i} = \straight{g'(r)}$ & No \\
        Newton's method & $\frac{e_{i+1}}{e_i^2} = \frac{f''(r)}{2f'(r)}$ or $\frac{e_{i+1}}{e_i} = \frac{m - 1}{m}$ & No \\
        Secant & $\frac{e_{i+1}}{e_i^p} = \frac{f''(r)}{2f'(r)}$ & No \\
        False position & ? & Yes
    \end{tabular}
\end{center}

$p = \frac{\sqrt{5} + 1}{2}$

\subsection*{Bisection}
Choose branch that preserves opposite sign.
\begin{align*}
    \epsilon &< \frac{b - a}{2^{n + 1}} \\
    n &= \ceiling{\log_2 \round{\frac{b - a}{\epsilon} - 1}}
\end{align*}

\subsection*{Fixed point}
\begin{align*}
    x_{i + 1} &= g(x)
\end{align*}

\subsection*{Newton's method}
\begin{align*}
    x_{i + 1} &= x_i - \frac{f(x_i)}{f'(x_i)}
\end{align*}

Simple root if $f'(x) \neq 0$, multiple root otherwise.

Multiplicity is first $n$ such $f^{(n)}(r) \neq 0$.

Stop when $\straight{x_{n+1} - x_n} < \epsilon$ or $\straight{f(x_n)} < \epsilon$.

\subsection*{Secant}
At each step, replace the older x with the new one.

\begin{align*}
    x_2 &= x_0 - \frac{f(x_0)\ (x_1 - x_0)}{f(x_1) - f(x_0)}
\end{align*}

\subsection*{False Position}
Same as secant method but bracketed. Pick the interval that makes the sign change.
\begin{align*}
    c &= a - \frac{f(a)\ (b - a)}{f(b) - f(a)}
\end{align*}

\subsection*{Fixed point}
To find roots:
\begin{align*}
    0 = f(x) \rightarrow x = g(x)
\end{align*}

To iterate:
\begin{align*}
    x_{i+1} &= g(x_i)
\end{align*}
Remember that we can also solve a root by solving for $x$ when the setting the formula equal to zero.

Converges when $\straight{g'(r)} > 1$.

\section*{Runtime}
Forward elimination or $A$ to $LU$:
\begin{align*}
    \frac{2}{3}n^3
\end{align*}

Backward substitution:
\begin{align*}
    n^2
\end{align*}

PLU with $n \times n$ matrix and $m$ equations to solve:
\begin{align*}
    \frac{2}{3} n^3 + 2 m n^2
\end{align*}

\begin{align*}
    \sum_{i=1}^n i &= \frac{1}{2}n (n + 1) \\
    \sum_{i=1}^n i^2 &= \frac{1}{6} n (n + 1) (2n + 1)
\end{align*}

\begin{align*}
    \vec{y} &= U\vec{x} \\
    L(U\vec{x}) &= \vec{b} \\
    L\vec{y} &= \vec{b} \\
    U\vec{x} &= \vec{y}
\end{align*}

\begin{align*}
    A\vec{x} &= \vec{b} \\
    PA\vec{x} &= P\vec{b} \\
    LU\vec{x} &= P\vec{b}
\end{align*}
Then solve the same with $P\vec{b}$ taking the role of $\vec{b}$.

$L$ can be saved in the compact form or represented as the product of elementary matrices for each row operation.

Remember $\text{first row cell} - \text{v} \times \text{pivot}$, where $v$ is the scalar to save in the cell of the $LU$ matrix.

\end{document}